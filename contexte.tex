\chapter{Contexte}


\section{Introduction}

%note en bas de page

La téléadministration, ou prise de contrôle à distance d'un ordinateur et de son système d'exploitation dans le but d’administrer le système (sauvegarde, mises à jour logiciel, etc.) et de résoudre les problèmes applicatifs des utilisateurs, est une solution bien adaptée aux PME \cite{ref4}. \\

Cette maintenance à distance consiste à prendre le contrôle via le réseau local ou Internet. Aujourd'hui, elle peut également être utilisée pour gérer un système d'exploitation virtualisé, hébergé en local ou également à distance. La télémaintenance se fait par le biais d'outils d'administration à distance qui proposent plus ou moins de fonctionnalités, comme le transfert de fichiers par exemple. Certains systèmes, comme KVM ou le port série, nécessitent un appareil connecté à la machine distante et pouvant être contrôlé depuis le poste de travail du technicien de maintenance.

    Les protocoles suivants sont ouverts et ont différentes applications ouvertes ou fermées : Commutateur écran-clavier-souris (ou commutateur KVM, indépendant du système), IPMI (indépendant du système, mais dépendant du matériel), Remote Desktop Protocol (ou RDP, indépendant du système), rsh (Remote Shell), Secure Shell (SSH) (ligne de commande ou export application X11 (ou X Window System), Telnet, XDMCP (X Window System) et VNC (indépendant du système), liaison série (généralement UART) compatible VT100 (console texte, indépendant du système, très utilisé dans les systèmes embarqués).

La plupart des systèmes d'exploitations comportent par défaut un ou plusieurs système de contrôle à distance, comme :

    RDP, telnet pour Microsoft Windows,
    RDP, telnet, rsh, SSH, UART (via TTY) VNC, XDMCP sur Linux ou BSD.

D'autres logiciels ont été développés par des sociétés tierces, permettent la télémaintenance, en passant généralement par des protocoles fermés ou l'un des protocoles ouverts définis ci-dessus : Bomgar, Checklan, NetOp Remote Control, DameWare, Ivanti Endpoint Manager, LogMeIn, PC Anywhere, TeamViewer, Ammyy Admin. 
Bla\\
%saut de paragraphe

Bla

\newpage

\section{Problématique soulevée}

Bla

\begin{center}
Problématique du sujet
\end{center}

\section{Hypothèse de solution}

%Quoi :
Bla\\

Voici une liste :
\begin{itemize}
\item item 1;
\item item 2;
\item item 3;
\item item 4.
\end{itemize}

Bla\\

%Comment :
Bla

Bla\footnotemark\\

%Detail :
Bla(cf. ref. \cite{cite6}).
%citation référencé dans le document "bibliographie.bib" inclus à la fin du document

\footnotetext{Note bas de page "bla"}


